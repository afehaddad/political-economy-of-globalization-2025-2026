\documentclass[10pt]{beamer}

% --- Packages ---
\usepackage[utf8]{inputenc}
\usepackage[T1]{fontenc}
%\usepackage[english]{babel}
\usepackage{amsmath, amssymb, booktabs, graphicx}
\usepackage{xcolor}
\usetheme[numbering=fraction,progressbar=frametitle]{metropolis}

% --- Couleurs ---
\definecolor{oxfordblue}{RGB}{0,33,71}
\definecolor{lightgray}{RGB}{245,245,245}
\setbeamercolor{frametitle}{bg=oxfordblue,fg=white}
\setbeamercolor{title}{fg=oxfordblue}
\setbeamercolor{progress bar}{fg=oxfordblue,bg=lightgray}
\setbeamerfont{frametitle}{series=\bfseries,size=\large}

% --- Informations ---
\title{Introduction: The Political Economy of Globalization}
\subtitle{Conceptual and Historical Perspectives}
\author{Ahmed Fouad El Haddad}
\institute{UPEC -- IEP Fontainebleau}
\date{\today}

% --- Début document ---
\begin{document}

% --- Page de titre ---
\begin{frame}
  \titlepage
\end{frame}

% --- Objectifs ---
\begin{frame}{Session Objectives}
\begin{itemize}
    \item Define \textbf{what globalization means} (Steger 2023).
    \item Introduce its \textbf{forms and imaginaries}.
    \item Trace its \textbf{historical trajectory} (O’Rourke 2022).
    \item Show why globalization must be studied through a \textbf{political economy lens}.
\end{itemize}
\end{frame}

% --- Concepts (Steger) ---
\begin{frame}{Key Concepts (Steger 2023)}
\begin{itemize}
    \item \textbf{Globalization}: expansion of social relations across world-space and 
world-time.
    \item \textbf{Globality}: condition of dense interconnections.
    \item \textbf{Global imaginary}: consciousness of the world as a single whole.
    \item \textbf{Globalisms}: ideologies interpreting globalization (neoliberal, 
justice-oriented, nationalist).
\end{itemize}
\end{frame}

% --- Forms (Steger) ---
\begin{frame}{Four Forms of Globalization}
\begin{columns}
\begin{column}{0.5\textwidth}
\begin{itemize}
    \item \textbf{Embodied}: flows of people (migration, refugees).
    \item \textbf{Disembodied}: ideas, information, digital data.
\end{itemize}
\end{column}
\begin{column}{0.5\textwidth}
\begin{itemize}
    \item \textbf{Objectified}: goods, commodities, capital.
    \item \textbf{Institutional}: organizations, states, empires, NGOs.
\end{itemize}
\end{column}
\end{columns}
\end{frame}

% --- Why political economy? ---
\begin{frame}{Why Political Economy of Globalization?}
\begin{block}{O’Rourke (2022)}
\begin{itemize}
    \item Globalization is not only technological $\rightarrow$ it is deeply 
\textbf{political}.
    \item Whether integration happens depends on \textbf{domestic politics and geopolitics}.
    \item Globalization produces \textbf{winners and losers} who mobilize politically.
\end{itemize}
\end{block}
\pause
\begin{itemize}
    \item Explains both \textbf{integration waves} and \textbf{periods of deglobalization}.
    \item Bridges globalization with \textbf{development}, \textbf{inequality}, and 
\textbf{conflict}.
\end{itemize}
\end{frame}

% --- Historical trajectory ---
\begin{frame}{Historical Trajectory (O’Rourke 2022)}
\begin{itemize}
    \item Prehistoric exchanges: obsidian, lapis lazuli (10,000 years ago).
    \item Bronze Age: copper, tin, trade institutions (weights, seals, money).
    \item Antiquity: Silk Roads, Indo-Roman trade, Rome–China links.
    \item Mongol Empire: Eurasian integration (13--14th c.).
    \item 1500s: Atlantic + Pacific circuits (Columbus, Manila galleon).
    \item 19th c.: \textbf{modern globalization} (steam, railroads, factor price convergence).
\end{itemize}
\end{frame}

% --- Reversibility ---
\begin{frame}{Globalization is Not Irreversible}
\begin{block}{Episodes of deglobalization}
\begin{itemize}
    \item Fall of Rome $\rightarrow$ collapse of long-distance trade.
    \item End of Mongol Empire $\rightarrow$ decline of caravan routes.
    \item Napoleonic Wars, WWI, WWII $\rightarrow$ disintegration of global trade.
    \item Cold War $\rightarrow$ systemic bifurcation (Soviet bloc).
\end{itemize}
\end{block}
\end{frame}

% --- Winners and losers ---
\begin{frame}{Winners and Losers}
\begin{itemize}
    \item 19th c.:  
    \begin{itemize}
        \item Europe: falling land rents, rising wages.  
        \item New World: rising land rents, falling wages.  
    \end{itemize}
    \item Parallel with today: “China shock”, Brexit, polarization.
    \item Political lesson: globalization $\rightarrow$ distributive conflicts $\rightarrow$ 
mobilization.
\end{itemize}
\end{frame}

% --- Development ---
\begin{frame}{Globalization and Development}
\begin{itemize}
    \item No region developed in isolation: economies evolved through \textbf{continuous 
interactions}.
    \item Industrial Revolution linked to:  
    \begin{itemize}
        \item Empire and colonial trade.  
        \item Slavery and plantation economies.  
        \item Circulation of knowledge and capital.  
    \end{itemize}
    \item Against “Whiggish” narratives: development was \textbf{contingent}, not inevitable.
\end{itemize}
\end{frame}

% --- Conclusion ---
\begin{frame}{Conclusion}
\begin{itemize}
    \item Globalization = \textbf{historical, political, reversible}.
    \item Steger: concepts, forms, imaginaries.  
    \item O’Rourke: history, reversibility, distribution, development.
    \item Political economy = the right lens to understand \textbf{integration and its 
discontents}.
\end{itemize}
\end{frame}

\end{document}

