\RequirePackage{etex}
\documentclass{beamer}
\usepackage{etex}
\usepackage{amsmath, amssymb, booktabs, graphicx}
\usepackage{xcolor}
\usepackage{pgfplots}
\usetikzlibrary{positioning}
\usepackage{hyperref}
\pgfplotsset{compat=1.17}

\usetheme{CambridgeUS}

% Pied de page élégant
\setbeamertemplate{footline}{%
  \leavevmode%
  \hbox{%
    \begin{beamercolorbox}[wd=\paperwidth,ht=2.25ex,dp=1ex,leftskip=1em,rightskip=1em]{author in head/foot}%
      \usebeamerfont{author in head/foot}%
      \insertshortauthor\hspace{1em}\hfill\insertshorttitle\hfill\insertframenumber/\inserttotalframenumber
    \end{beamercolorbox}}%
  \vskip0pt%
}

\title{The Political Economy of Globalization}
\author{Ahmed Fouad EL HADDAD}
\institute{UPEC -- IEP Fontainebleau}
\date{\today}

\begin{document}

% --- Page de titre ---
\begin{frame}
    \titlepage
\end{frame}

% --- Présentation de l'intervenant ---
\begin{frame}{About the Lecturer}
\centering
\begin{itemize}
    \item \textbf{Name:} Ahmed Fouad El Haddad  
    \item \textbf{Position:} Associate Professor of Political Science, IEP Fontainebleau – Université Paris-Est Créteil  
    \item \textbf{Research Areas:} Comparative politics, authoritarian political economy, social policies, computational social sciences  
\end{itemize}
\end{frame}

% --- Introduction par l'iPhone ---
\begin{frame}{A Familiar Object}
\centering
\includegraphics[width=0.4\textwidth]{iphone.jpg}

\vspace{0.3cm}
\textbf{The iPhone as a window into globalization:}
\begin{itemize}
    \item Designed in California.
    \item Assembled in China.
    \item Components sourced worldwide.
    \item Sold and used globally.
\end{itemize}
\end{frame}

\begin{frame}{Global Value Chains and Uneven Globalization}

\small

\begin{itemize}
    \item Globalization can look \textbf{beautiful} — innovation, connectivity, access to goods and information.
    \item It can also look \textbf{ugly} — exploitation, inequality, environmental damage, and e-waste.
    \item Your experience depends on \textbf{where you stand in the global value chain}.
\end{itemize}

\vspace{0.5em}
Most people engage with globalization through everyday products — smartphones, clothing, food —  
but rarely consider how these are made.

\vspace{1em}
\centering
\includegraphics[width=0.5\linewidth]{chaine.png}

\vspace{0.3em}
\scriptsize \textit{A typical global value chain: extraction → assembly → consumption → disposal.}

\end{frame}


\begin{frame}{Who Stands Where in the Global Chain?}

\small
% --- First Row: text and block side by side ---
\begin{columns}[T,onlytextwidth]

    % LEFT: explanatory text
    \begin{column}{0.65\textwidth}
        If you own an iPhone, you're likely at the \textbf{consumption end} of the chain — benefiting from convenience, design, and global access.

        Others, however, are positioned on more precarious links in the chain:

        \begin{itemize}
            \item \textbf{Extraction} — cobalt miners in the Congo working under dangerous conditions.
            \item \textbf{Assembly} — factory workers in Shenzhen performing repetitive labor for low wages.
            \item \textbf{Disposal} — e-waste handlers in Ghana or India exposed to toxic components.
        \end{itemize}
    \end{column}

    % RIGHT: block argument
    \begin{column}{0.35\textwidth}
        \begin{block}{Core Argument}
        Globalization is a \textbf{distributional conflict} over resources, labor, and opportunity.  
        It must be understood in both \textbf{economic} and \textbf{power-relational} terms.
        \end{block}
    \end{column}

\end{columns}
\end{frame}

% --- Winners and losers ---
\begin{frame}{Winners and Losers of the iPhone Economy}
\small
\begin{table}[h!]
\centering
\renewcommand{\arraystretch}{1.3}
\begin{tabular}{p{3cm} p{3.2cm} p{3.2cm}}
\toprule
\textbf{Dimension} & \textbf{Winners} & \textbf{Losers} \\
\midrule
Geography & Apple (USA), high-tech suppliers (Japan, Korea, Taiwan) & Assembly workers in China (low wages, harsh conditions) \\
\midrule
Capital & Shareholders, global investors & Local firms excluded from supply chain \\
\midrule
Environment & Consumers enjoying digital connectivity & Mining regions, e-waste sites \\
\bottomrule
\end{tabular}
\end{table}

\begin{block}{Lesson}
The iPhone generates \textbf{profits and innovation}, but also \textbf{inequalities and conflicts}.  
This is why we need a \textbf{political economy perspective}.
\end{block}
\end{frame}

\begin{frame}{From iPhone to Obsidian: A Long History of Globalization}

\small
\begin{itemize}
    \item Globalization is not new. Long-distance trade predates modern capitalism.
    \item It is not linear: periods of intense connection alternate with deglobalization.
    \item It is not just economic: globalization reshapes culture, identity, power.
    \item It is not neutral: globalization produces \textbf{winners and losers}.
\end{itemize}

\vspace{0.6em}
\begin{block}{Key Idea (O’Rourke \& Williamson, 2019)}
Globalization is shaped not only by technology, but by \textbf{political economy forces} — from domestic institutions to geopolitical conflict.
\end{block}

\vspace{0.4em}
\scriptsize
\textit{→ To understand globalization, we need theory, history, and critical thinking.}
\end{frame}

% --- What is Political Economy ---
\begin{frame}{What Do We Mean by Political Economy?}

\begin{columns}[T] % Top align both columns
    % TEXT COLUMN
    \begin{column}{0.65\textwidth}
        \begin{itemize}
            \item \textbf{Political Economy (PE)} — the study of how \textbf{economic processes} are shaped by \textbf{power}, \textbf{institutions}, and \textbf{conflict}.
            
            \item \textbf{International Political Economy (IPE)} — focuses on how global flows of trade, finance, and production intersect with \textbf{geopolitics} and \textbf{domestic politics}.

            \item \textbf{This course:} we use the tools of PE and IPE to understand globalization’s:
            \begin{itemize}
                \item Historical trajectories
                \item Winners and losers
                \item Political consequences
            \end{itemize}
        \end{itemize}
    \end{column}

    % IMAGE COLUMN
    \begin{column}{0.3\textwidth}
        \begin{minipage}{\linewidth}
            \includegraphics[width=\linewidth]{Eco2.png}
            \centering\small (a)
        \end{minipage}
        \vspace{0.5em}
        
        \begin{minipage}{\linewidth}
            \includegraphics[width=\linewidth]{eco1.jpg}
            \centering\small (b)
        \end{minipage}
    \end{column}
\end{columns}

\end{frame}



\begin{frame}{From Political Economy to Global Studies}
\small
\begin{itemize}
    \item Political Economy and International Political Economy (IPE) have helped us understand how economics and politics interact — nationally and globally.
    
    \item However, globalization cannot be fully explained through nation-states or economic structures alone.

    \item Globalization is messy, multidimensional, and often contradictory. It compresses time and space, connects and disconnects, empowers and marginalizes.

    \item Scholars from different disciplines disagree not only on the causes or consequences of globalization, but even on what it actually is.

    \item This intellectual complexity opened the way for a new academic response: \textbf{Global Studies}.
\end{itemize}

\centering
\textit{What happens when globalization becomes the object of study $\Rightarrow$ not just a background condition?}

\end{frame}

\begin{frame}{The Rise of Global Studies}

\small
\begin{itemize}
    \item Since the 1990s, \textbf{Global Studies} emerged as a new academic response to the challenge of understanding globalization.

    \item Unlike traditional disciplines — which center on concepts like ‘society’, ‘power’, or ‘scarcity’ — Global Studies places \textbf{globalization itself} at the center.

    \item It rejects \textit{methodological nationalism} (IR’s focus on states) in favor of \textbf{methodological globalism} — focusing on flows, networks, and non-state actors.

    \item Global Studies emphasizes:
    \begin{itemize}
        \item Multidimensionality: economy, culture, ecology, technology, ideology...
        \item Interdisciplinarity: bridging social sciences, humanities, and beyond
        \item Fluidity: dynamic and contested processes
        \item Critical thinking: questioning power, inequality, and exclusion
    \end{itemize}

    \item Today, it structures hundreds of academic programs worldwide.
\end{itemize}
\end{frame}

\begin{frame}{What Is Globalization? Beyond Flows and Borders}

\small


% --- Part 2: Steger's concepts ---
\textbf{Key conceptual distinctions (Steger, 2023):}
\begin{itemize}
    \item \textbf{Globalization} — expansion of social relations across world-space and world-time.
    \item \textbf{Globality} — condition of dense, multi-scalar interconnections.
    \item \textbf{Global imaginary} — growing awareness of the world as a single space.
    \item \textbf{Globalisms} — competing ideologies about what globalization means (e.g., neoliberal, justice-oriented, nationalist).
\end{itemize}

\end{frame}

\begin{frame}{Broader understanding of globalization}
    

% --- Part 1: Broader understanding of globalization ---
\textbf{Globalization is not only about trade and migration:}
\begin{itemize}
    \item It transforms \textbf{identities}, \textbf{imaginaries}, and \textbf{institutions}.
    \item It connects distant places into a \textbf{single interdependent system}.
    \item It produces new forms of \textbf{inequality}, but also \textbf{interdependence}.
    \item It operates simultaneously on \textbf{economic}, \textbf{social}, \textbf{cultural}, and \textbf{political} levels.
\end{itemize}
\end{frame}


% --- Four Forms of Globalization (Steger) ---
\begin{frame}{Four Forms of Globalization}

\begin{columns}[c,onlytextwidth]

    % LEFT COLUMN: Content description
    \begin{column}{0.52\textwidth}
        \begin{itemize}
            \setlength\itemsep{0.9em}
            \small
            \item \textbf{Embodied} — Movement of people: migrants, tourists, refugees. Human bodies act as carriers of ideas, labor, and culture.

            \item \textbf{Disembodied} — Transmission of information, ideas, and digital content. Flows occur via media, the internet, and communications.

            \item \textbf{Objectified} — Global movement of goods, commodities, and capital. Includes material trade and financial exchange.

            \item \textbf{Institutional} — Operations of global structures: states, NGOs, empires, and international organizations.
        \end{itemize}
    \end{column}

    % RIGHT COLUMN: Visual representation
    \begin{column}{0.5\textwidth}
        \centering
        \includegraphics[width=\linewidth]{types.pdf}
        \vspace{0.3em}

        \scriptsize
        \textbf{Figure:} The four forms of globalization.\\
        \textit{Source:} M. B. Steger, \textit{Globalization: A Very Short Introduction}, 6th ed. (2024).
    \end{column}

\end{columns}

\end{frame}

\begin{frame}{The iPhone and the Forms of Globalization}

\small

\begin{columns}[T,onlytextwidth]

    % LEFT COLUMN
    \begin{column}{0.58\textwidth}
        \textbf{The iPhone illustrates four forms of globalization (Steger, 2024):}
        \begin{itemize}
            \item \textbf{Embodied:} Migrant labor in factories (e.g., Foxconn in China).
            \item \textbf{Disembodied:} Apple’s design, branding, and digital ecosystem.
            \item \textbf{Objectified:} Physical components sourced and shipped worldwide.
            \item \textbf{Institutional:} Trade rules, tax regimes, and supply chain governance.
        \end{itemize}

        \vspace{0.5em}
        \textit{The iPhone condenses globalization into a single consumer object.}
    \end{column}

\end{columns}
\end{frame}


\begin{frame}{Course Overview: Political Economy of Globalization}

\small
\textbf{Objective:} Equip you with conceptual and analytical tools to understand globalization as a political-economic process.

\vspace{0.5em}
\begin{columns}[T,onlytextwidth]
    \begin{column}{0.48\textwidth}
        \textbf{Foundations:}
        \begin{itemize}
            \item What is globalization?
            \item States vs. markets (Polanyi, Marx)
            \item Varieties of capitalism
        \end{itemize}

        \textbf{Structures:}
        \begin{itemize}
            \item Global governance
            \item Regimes and institutions
            \item Dependency and world-systems
        \end{itemize}
    \end{column}

    \begin{column}{0.48\textwidth}
        \textbf{Crises and Conflicts:}
        \begin{itemize}
            \item Globalization and inequality
            \item Crises and deglobalization
            \item Populism, backlash, and protectionism
        \end{itemize}

        \textbf{Futures:}
        \begin{itemize}
            \item Deglobalization?
            \item Technology and climate
            \item China and the new world order
        \end{itemize}
    \end{column}
\end{columns}

\vspace{0.5em}
\centering
\scriptsize
\textit{→ 10 sessions to analyze globalization critically — from theory to current debates.}
\end{frame}



\end{document}
